\ifbool{printVersion}{
	\cleardoublepage % ak to vychadza na parnu stranu, vlozi sa prazdna strana
}

\section*{Abstrakt}

\noindent \textbf{Názov práce:} Modelovanie rozdelenia pravdepodobnosti zmesi spojitých a diskrétnych náhodných premenných\\
\textbf{Abstrakt:} Táto práca sa venuje návrhu a implementácii všeobecného rámca pre modelovanie združeného rozdelenia pravdepodobnosti dvojice náhodných premenných, ktoré môže byť spojité, diskrétne alebo zmiešané. Cieľom je vytvoriť systém, ktorý dokáže efektívne reprezentovať štatistickú štruktúru takýchto dát a umožniť následné odvodenie podmienených rozdelení využiteľných v regresii a klasifikácii. Práca zahŕňa teoretický prehľad základných prístupov k modelovaniu jednorozmerných aj viacrozmerných rozdelení, vrátane rozkladu na marginálne rozdelenia a kopulové funkcie. Kľúčovú časť tvorí implementácia rôznych modelovacích funkcií v jazyku R – ako pre čisto diskrétne, tak aj spojité či zmiešané vektory. Tieto modely sú založené na parametrických, neparametrických a hybridných prístupoch (napr. KDE, normálne rozdelenie, t-rozdelenie). Záver práce tvorí prezentácia praktických aplikácií týchto modelov na reálnych dátach ako aj ukážka interaktívnej aplikácie. Výsledkom je nástroj, ktorý umožňuje detailnú štatistickú analýzu dát s homogénnou aj heterogénnou štruktúrou, a zároveň otvára priestor pre ďalšie rozšírenia smerom k viacrozmerným modelom a efektívnejšiemu modelovaniu zmiešaného vektora.

\vspace{10pt}

\noindent \textbf{Kľúčové slová:} zmiešaný náhodný vektor, štatistické modelovanie, združené rozdelenie, podmienená hustota, aplikácia

\vspace{+20pt}


\section*{Abstract}
\noindent \textbf{Title:} Modelling joint probability distribution of a mixture of continuous and categorical random variables\\
\textbf{Abstract:} Preklad abstraktu do angličtiny (poriadny, nie Google Translate). Je dobré ho robiť až keď je autor (a školiteľ) spokojný s tým slovenským, aby ste ho zbytočne nemuseli prekladať niekoľkokrát.

\vspace{10pt}

\noindent \textbf{Keywords:} 3 až 5 kľúčových slov/slovných spojení oddelených čiarkou
