\chapter{Záver}

Cieľom tejto bakalárskej práce bolo navrhnúť a implementovať ucelený rámec pre modelovanie \textit{združeného rozdelenia pravdepodobnosti} dvoch náhodných premenných, pričom sa zohľadňovali všetky typové kombinácie: \textit{diskrétne}, \textit{spojité} a \textit{zmiešané} rozdelenia. Sústredili sme sa nielen na samotné odhady hustôt a pravdepodobnostných funkcií, ale aj na odvodenie a vizualizáciu podmienených rozdelení, ktoré zo združeného modelu vyplývajú.

V práci sa podarilo naplniť všetky hlavné ciele. Boli implementované robustné funkcie pre modelovanie diskrétnych rozdelení pomocou PMF, spojitých rozdelení prostredníctvom parametrických a neparametrických prístupov (vrátane rozkladu cez kopulu), ako aj zmiešaných rozdelení cez vážený súčet podmienených hustôt vzhľadom na kategórie diskrétnej premennej. Následne bola implementovaná logika pre výpočet podmienených hustôt – osobitne pre spojitú aj diskrétnu odozvu. Všetky modely boli navrhnuté s dôrazom na konzistenciu výstupov, modularitu a možnosti vizualizácie a sumarizácie výsledkov.

Silnou stránkou riešenia je flexibilita – používateľ si môže zvoliť typ modelu (neparametrický, normálny, t-rozdelenie), typ rozkladu prostredníctvom výberu typu kopule a modelu pre marginálne hustoty, a získať nielen výpočtové výstupy, ale aj zrozumiteľné vizualizácie. Za výhodu možno považovať aj to, že architektúra implementácie je rozšíriteľná – nové typy distribúcií alebo alternatívne spôsoby modelovania by sa dali pridať relatívne jednoducho.

Na druhej strane, určité obmedzenia vznikajú pri malých vzorkách, kde niektoré modely (najmä KDE alebo t-rozdelenie) môžu generovať nespresnené odhady alebo vyžadujú špeciálne ošetrenie. Taktiež použitie výlučne dvojrozmerných modelov je síce didakticky vhodné, no v praxi by sa mnohé úlohy dali riešiť iba rozšírením do vyšších dimenzií.

V budúcnosti by bolo možné prácu rozšíriť niekoľkými smermi. Prvým je napríklad zavedenie viacrozmerných kopulových modelov a ich porovnávanie pomocou metód selekcie modelu (napr. AIC, BIC). Ďalším prirodzeným krokom by bolo implementovať logiku modelovania združenej hustoty pravdepodobnosti zmiešaného vektora cez vážený súčet podmienených hustôt s tým, že by bolo možné pre každú takúto hustotu použiť rozdielny typ modelu, čo by mohlo zabezpečiť presnejší opis rozdelenia pravdepodobnosti tohto typu a väčšiu flexibilitu modelu. Taktiež by bolo možné implementovať tzv. \textit{conditional copula modeling} metódu, ktorá by nám umožňovala modelovať každú podmienenú hustotu ako osobitnú štruktúru za pomoci kopuly a marginálnych rozdelení, čo opäť môže byť veľmi efektívny spôsob modelovania takejto zmiešanej štruktúry.

Výsledky prezentované v tejto práci vytvárajú pevný základ pre analytické aj praktické využitie modelovania združených rozdelení ako účinný analytický nástroj.