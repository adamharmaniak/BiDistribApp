\chapter{V akom editore písať? Aký kompilátor používať?}\label{sec:aky_editor}

\subsubsection{Editor}
Existuje niekoľko populárnych LaTeX editorov: TeXstudio, TeXmaker, TeXnicCenter, LyX, Overleaf (ten je online),... Dobrý prehľad nájdete \href{https://beebom.com/best-latex-editors/}{napríklad tu}. \textbf{Odporúčam TeXstudio}, dá sa v ňom pracovať veľmi efektívne. V akomkoľvek editore sa naučte \textbf{používať klávesové skratky} (v TeXstudiu ich nájdete v Options/Configure TeXstudio/Shortcuts), rýchlo sa orientovať a preklikávať v TeX súbore pomocou referencií, preklikávať sa medzi TeX súborom a pdf, neuveriteľne to zrýchli prácu.

Ak sa vám páči mať vo veciach systém a poriadok, tak dobrý nápad je priebežne prácu commitovať do repozitára na \href{https://github.com/}{GitHub}-e. Mimochodom, GitHub je veľmi užitočný pri programovaní -- hlavne ak s niekým spolupracujete (napríklad so školiteľom) alebo píšete komplexnejší program (v rámci záverečnej práce to tak často býva). Ak GitHub nepoužívate, odporúčam začať.

\subsubsection{Kompilátor}
Akýkoľvek editor si zvolíte, tak na kompilovanie odporúčam \textbf{kompilátor XeLaTeX}. Výsledné PDF bude mať správne kódovanie slovenských znakov\footnote{Vďaka tomu, že XeLaTeX používa Unicode.} (v texte sa bude dať dobre vyhľadávať). Pre väčšinu editorov (aj TeXstudio) sa dá kompilátor nastaviť aj priamo v tex súbore pomocou tzv \uv{magic commentu}, v súbore \verb|_Zaverecna_praca.tex| je to riadok \verb|% !TeX program = xelatex|, ktorý editoru hovorí, že tento dokument má kompilovať pomocou kompilátora XeLaTeX. Samozrejme, v TeXstudiu je možné aj prenastaviť defaultný kompilátor pre všetky dokumenty: Options/Configure TeXstudio/Build/Default compiler/XeLaTeX.

Ak ale chcete kompilovať pomocou \textbf{kompilátora PdfLaTeX}, treba správne kódovanie v PDF zabezpečiť pomocou pridania balíkov do preambuly. Funguje napríklad táto kombinácia (ale sú aj iné možnosti):
\begin{verbatim}
	\usepackage[utf8]{inputenc} % input encoding
	\usepackage[T1]{fontenc} % spravne makcene
	\usepackage{lmodern} % Latin modern fonts
\end{verbatim}
V súbore \verb|_Zaverecna_praca.tex| treba vymazať magic comment \verb|% !TeX program = xelatex| (ak defaultne používate PdfLaTeX) resp. ho prepísať na \verb|% !TeX program = pdflatex|.

\subsubsection{Porovnávanie verzií (diff)}
Keď školiteľovi posielate novú verziu, je veľmi užitočné mu v nej vyznačiť zmeny oproti poslednej verzii, ktorú čítal. Odporúčam na to výborný skript \textbf{latexdiff} napísaný v jazyku Perl. Je to analógia funkcie Track changes vo Worde, ktorú možno poznáte. Skript sa inštaluje napríklad takto:
\begin{enumerate}
	\item Nainštalujte latexdiff. Napríklad cez MiKTeX Console (Run as administrator), Packages.
	\item Nainštalujte \href{https://strawberryperl.com/}{Strawberry Perl}.
\end{enumerate}
Keď už máte skript nainštalovaný, používa sa veľmi jednoducho a rýchlo (keď už sa to naučíte):
\begin{enumerate}
	\item V príkazovom riadku sa nastavíte do správneho priečinka (rýchly spôsob v TeXstuidu je Tools/Open External Terminal, a tam prípadne príkaz \verb|cd ..| na posun o priečinok vyššie).
	\item V príkazovom riadku treba spustiť príkaz typu
	\begin{verbatim}
		latexdiff --flatten "stara_verzia.tex" "nova_verzia.tex" > "diff.tex"
	\end{verbatim}
	Vstupom je stará verzia dokumentu \verb|stara_verzia.tex|, nová verzia \verb|nova_verzia.tex| a výstupom je súbor \verb|diff.tex|, v ktorom sú vyznačené zmeny. Nastavenie \verb|--flatten| je potrebné, keď sa dokument skladá z viacerých súborov (ako táto šablóna). Použitie skriptu na túto šablónu môže vyzerať takto
	\begin{verbatim}
		latexdiff --flatten "SK_verzia_stara\_Zaverecna_praca.tex" "SK_verzia\
		_Zaverecna_praca.tex" > "Porovnanie_verzii\diff.tex"
	\end{verbatim}
	V príkazovom riadku teda musíte byť v priečinku \verb|ZaverecnaPracaMPM\praca|, v ktorom sa nachádza priečinok so starou verziou\footnote{Na uchovávanie starých verzií aj tu výborne poslúži GitHub, ale ak ho nepoužívate, tak si priebežne zálohujte celý priečinok s prácou.} \verb|SK_verzia_stara| aj s novou verziou \verb|SK_verzia| a musí byť vytvorený priečinok \verb|Porovnanie_verzii|. Ak máte iné umiestnenia, cesty si upravte. Pokiaľ nemáte v názvoch priečinkov a súborov medzery, tak úvodzovky v príkazoch nie sú potrebné.
	\item Do priečinka \verb|Porovnanie_verzii| dajte všetky obrázky a dokumenty (podpriečinky \verb|figures| a \verb|documents|), ktoré sa do práce vkladajú a aj bibliografický \verb|.bib| súbor (všetko z novej verzie).
	\item Súbor \verb|diff.tex| skompilujete, čím vytvoríte výsledné \verb|diff.pdf|, ktoré môžete spolu s aktuálnou verziou poslať školiteľovi.
\end{enumerate}







